In class we discussed surface integrals without regard to orientation. By reparameterizing the surface integral could be written as 
\begin{equation*}
I=\int \int_s G\left(s\right)dS=\int \int_R G\left(r\left(u,v\right)\right)\vert N\left(u,v\right)\vert dudv
\end{equation*}

\begin{enumerate}
	\item Consider the case $G=z$ adn teh surface $S$ is the hemisphere $x^2+y^2+z^2=9$ with $z\geq0$. Use polar coordinates and evaluate the right hand side of the above reult. 
	\item The surface $S$ is also given explicitly by $z=f\left(x,y\right)=\sqrt{9-x^2-y^2}$. For such cases the surface integral can be rewritten as 
	\begin{equation}
		\int \int_S G\left(r\right)dA=\int \int_{R*} G(x,y,f(x,y))\sqrt{1+\left(\frac{\partial f}{\partial x}\right)^2+\left(\frac{\partial f}{\partial y}\right)^2}dxdy
		\end{equation}
		Evaluate the right-hand side fo this result.
		
\end{enumerate}

\begin{enumerate}
	\item Rewriting the function in polar coordinates yields: $r^2\cos^2\theta+r^2\sin^2\theta+z^2=9$

	\item 
\end{enumerate}
			