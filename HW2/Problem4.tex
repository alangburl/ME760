Find the spectra and eigenvectors for the two matrixes below. Show you work.
\begin{equation*}\boldsymbol{A}=
	\begin{pmatrix}
		3&5&3\\0&4&6\\0&0&1
	\end{pmatrix}
\end{equation*}
\begin{equation*}
	\boldsymbol{A}=
	\begin{pmatrix}
		a&1&0\\1&a&1\\0&1&a
	\end{pmatrix}
\end{equation*}

The spectra of each matrix is defined as the set of its eignevalues. Calculation of the eignevalues is realized through teh characteristic equation and found by equation \ref{eq:eigenvalues}.
\begin{equation}
	\vert \boldsymbol{A}-\lambda\boldsymbol{I}\vert=0
	\label{eq:eigenvalues}
\end{equation}
Substituting values in and calculation of the determinate yields the following characteristic equations abd spectra:
\begin{equation*}
	\left(3-\lambda\right)\left(4-\lambda\right)\left(1-\lambda\right)=0
\end{equation*}
with a spectra of:
\begin{equation*}
	\boxed{
	\lambda\left(\boldsymbol{A}\right)=
	\begin{Bmatrix}
		3&4&1
	\end{Bmatrix}}
\end{equation*}
Finding the determinate and expanding the characteristic polynomial yields:
\begin{equation*}
	\left(a-\lambda\right)^3-2\left(a-\lambda\right)=0
\end{equation*}
Factoring this yields the following solutions:
	\begin{equation*}
	\boxed{
	\lambda\left(\boldsymbol{A}\right)=
		\begin{Bmatrix}
			a& a\pm\sqrt{2}
		\end{Bmatrix}}
	\end{equation*}
	
Eigenvector for each of the associated eigenvalues must satisfy:
\begin{equation}
	\boldsymbol{A}\boldsymbol{v}=\lambda\boldsymbol{v}
\end{equation}
Rearrangment to solve for $\boldsymbol{v}$:
\begin{equation}
	\left(\boldsymbol{A}-\lambda\boldsymbol{I}\right) \cdot \boldsymbol{v}=0
	\label{eq:vec}
\end{equation}
Each of the aforementioned eigenvalues are substituted into equation \ref{eq:vec} and solve yields the non-unique eigenvectors:
\begin{enumerate}
	\item $\lambda=1$
		\begin{equation*}
		\left(\boldsymbol{A}-\lambda\boldsymbol{I}\right)=
		\begin{pmatrix}
			2&5&3\\0&3&6\\0&0&0
		\end{pmatrix}
		\end{equation*}

As can be seen, $\boldsymbol{v}$ is underdeterminate, as such any value for $\boldsymbol{v}_3$ can be used, 2 is chosen. Using this and substituiting into the other equation yields:
	\begin{equation*}
	\boxed{
			\boldsymbol{v}\left(\lambda=1\right)=
		\begin{Bmatrix}
			7\\-4\\2
		\end{Bmatrix}}
	\end{equation*}

\item $\lambda=3$
		\begin{equation*}
		\left(\boldsymbol{A}-\lambda\boldsymbol{I}\right)=
		\begin{pmatrix}
			0&5&3\\0&1&6\\0&0&-2
		\end{pmatrix}
		\end{equation*}
Again, the system is underdeterminate, thus $\boldsymbol{v}_1=$ is chosen for simplicity, resulting in the following eigenvector
	\begin{equation*}
	\boxed{
			\boldsymbol{v}\left(\lambda=3\right)=
		\begin{Bmatrix}
			1\\0\\0
		\end{Bmatrix}}
	\end{equation*}
\item $\lambda=4$
		\begin{equation*}
		\left(\boldsymbol{A}-\lambda\boldsymbol{I}\right)=
		\begin{pmatrix}
			-1&5&3\\0&0&6\\0&0&-3
		\end{pmatrix}
		\end{equation*}
It can clearly be seen this system is underdeterminate, as such $\boldsymbol{v}_2=1$ is chosen for simplicity. With the resulting eigenvector of:
	\begin{equation*}
	\boxed{
			\boldsymbol{v}\left(\lambda=4\right)=
		\begin{Bmatrix}
			5\\1\\0
		\end{Bmatrix}}
	\end{equation*}
\end{enumerate}
Similiar to the previous problem, each eignevalue is substituted in the associated eigenvector
\begin{enumerate}
\item $\lambda=a$
		\begin{equation*}
		\left(\boldsymbol{A}-\lambda\boldsymbol{I}\right)=
		\begin{pmatrix}
			0&1&0\\1&0&1\\0&1&0
		\end{pmatrix}
		\end{equation*}
Choosing $\boldsymbol{v}_1=1$, the eigenvector becomes:
		\begin{equation*}
	\boxed{
			\boldsymbol{v}\left(\lambda=a\right)=
		\begin{Bmatrix}
			1\\0\\-1
		\end{Bmatrix}}
	\end{equation*}
	
\item $\lambda=a\pm\sqrt{2}$
		\begin{equation*}
		\left(\boldsymbol{A}-\lambda\boldsymbol{I}\right)=
		\begin{pmatrix}
			\pm\sqrt{2}&1&0\\1&\pm\sqrt{2}&1\\0&1&\pm\sqrt{2}
		\end{pmatrix}
		\end{equation*}
		Simplification of both eignevalue solutions yields a trivial eigenvector of:
				\begin{equation*}
	\boxed{
			\boldsymbol{v}\left(\lambda=a\pm\sqrt{2}\right)=
		\begin{Bmatrix}
			0\\0\\0
		\end{Bmatrix}}
	\end{equation*}
\end{enumerate}