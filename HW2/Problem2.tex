In practical work the following formulat is quite useful. 
\begin{equation}
	\vert\vec{a} \times \vec{b} \vert= \sqrt{\left(\vec{a}\cdot\vec{a}\right) \left(\vec{b}\cdot\vec{b}\right)-\left(\vec{a}\cdot\vec{b}\right)^2}
	\end{equation}
Give a proof.\\ 

By definition $\vec{a} \times \vec{b}=\|a\|\|b\|\sin \theta$ and $\vec{a}\cdot\vec{b}=\|a\|\|b\|\cos\theta$
Substituting the above definitions and squaring both sides produces:
\begin{equation*}
	\|a\|^2\|b\|^2\sin^2 \theta=\|a\|^2\|b\|^2-\|a\|^2\|b\|^2\cos^2\theta
\end{equation*}

Rearranging and factoring to utilize trigonometric identities:
\begin{equation*}
	\|a\|^2\|b\|^2\left(\sin^2 \theta+\cos^2\theta\right)=\|a\|^2\|b\|^2
\end{equation*}
Using $\sin^2 \theta+\cos^2\theta=1$, the equations are equivalent.
\begin{equation*}
	\|a\|^2\|b\|^2=\|a\|^2\|b\|^2
\end{equation*}