Prove taht trace of a square real of complex matrix A equals the sum of its eigenvalues. This fact is often useful check on the accuracy of eigenvalue calculations. Demonstrate with an example of your choosing.


Proof:
For any square matrix there exists a matrix $\boldsymbol{P}$ such that $\bodsymbol{J}=\boldsymbol{P}\boldsymbol{A}\boldsymbol{P^{-1}}$ where $\boldsymbol{J}$ is Jordan Conical form. 

Example: (Using matrix from probelm 6)
The eigenvalues can be found from the characteristic polynomial:
\begin{equation*}
	\left(\lambda-16\right)\left(\lambda+8\right)\left(\lambda-4\right)=0
\end{equation*}
Giving rise to the eigenvalues to be equal to $\left[16, -8, 4\right]$. The sum of which is $12$
The trace of a matrix is the sum of the diagonals, for which $tr\left(A\right)=12$.
