For the array
\begin{equation*}
C=
\begin{pmatrix}
	4 & 6&2 \\6 &0&3\\2&3&-1
\end{pmatrix}
\end{equation*}
calculate:
\begin{enumerate}
	\item $\boldsymbol{C}^2$
	\item $\boldsymbol{C}^T\boldsymbol{C}$
	\item $\boldsymbol{C}\boldsymbol{C}^T$
\end{enumerate}
\begin{enumerate}
	\item Each element in the product can be expressed from equation \ref{eq:mat_mul} where $i$ and $j$ are the row and column indices respectively. 
	\begin{equation}
		\boldsymbol{C}^2_{i,j}=\sum_{n=1}^{3}C_{i,n}C_{n,j}
		\label{eq:mat_mul}
	\end{equation}
The subsuquent calculations are seen below:
	\begin{align*}
		\boldsymbol{C}^2_{1,1}=4*4+6*6+2*2 =56\\
		\boldsymbol{C}^2_{1,2}=4*6+6*0+2*3=30\\
		\boldsymbol{C}^2_{1,3}=4*2+6*3+2*-1=24\\
		\boldsymbol{C}^2_{2,1}=6*4+0*6+2*2=30\\
		\boldsymbol{C}^2_{2,2}=6*6+0*0+3*3=45\\
		\boldsymbol{C}^2_{2,3}=6*2+0*3+3*-1=9\\
		\boldsymbol{C}^2_{3,1}=2*4+3*6+-1*2=24\\
		\boldsymbol{C}^2_{3,2}=2*6+3*0+-1*3=9\\
		\boldsymbol{C}^2_{3,3}=2*2+3*3+-1*-1=14
	\end{align*}
The resulting matrix therefore is:
\begin{equation*}
\boxed{
\boldsymbol{C}^2=
\begin{bmatrix}
	56 &30&24\\30&45&9\\24&9&14
\end{bmatrix}}
\end{equation*}

\item Calculation of the transpose, $\boldsymbol{C}^T$ of $\boldsymbol{C}$ is realized through equation \ref{eq:transpose}.
\begin{equation}
	C^T_{i,j}=C_{j,i}
	\label{eq:transpose}
\end{equation}
Computation of $\boldsymbol{C}^T$, results in $\boldsymbol{C}^T=\boldsymbol{C}$. Thus the calculation of $\boldsymbol{C}\boldsymbol{C}^T$ is identical to that of (a). Thus the resulting matrix is:
\begin{equation*}
\boxed{
\boldsymbol{C}^T\boldsymbol{C}=
\begin{bmatrix}
	56 &30&24\\30&45&9\\24&9&14
\end{bmatrix}}
\end{equation*}
\item Similiar to (b), the calculation of $\boldsymbol{C}\boldsymbol{C}^T$ is identical to $\boldsymbol{C}^2$. Thus the resulting matrix is:
\begin{equation*}
\boxed{
\boldsymbol{C}\boldsymbol{C}^T=
\begin{bmatrix}
	56 &30&24\\30&45&9\\24&9&14
\end{bmatrix}}
\end{equation*}
\end{enumerate}
	
