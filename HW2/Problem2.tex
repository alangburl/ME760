Solve $\boldsymbol{A}\boldsymbol{x}=\boldsymbol{b}$ for the following set of linear equations:
\begin{equation*}
	\begin{pmatrix}
		1&1&-1\\0&8&6\\-2&4&-6
	\end{pmatrix}
	\begin{pmatrix}
		x\\y\\z
	\end{pmatrix}
	=
	\begin{pmatrix}
		9\\-6\\40
	\end{pmatrix}
\end{equation*}
	\begin{enumerate}
		\item by Gauss Elimination
		\item by using Cramer's Rule
		\item by finding the inverse $\boldsymbol{x}=\boldsymbol{A}^{-1}\boldsymbol{b}$
	\end{enumerate}

\begin{enumerate}
	\item The goal of Gaussian Elimination is to represent the system in reduced row echelon form (RREF), the following steps are used to reduce $\boldsymbol{A}$ to RREF:
	
	\begin{enumerate}
		\item Divide row 2 by 8:
			\begin{equation*}A=
				\begin{pmatrix}[ccc|c]
					1&1&-1&9\\0&1&0.75&-0.75\\-2&4&-6&40
			\end{pmatrix}
			\end{equation*}

		\item Multiply row 1 by 2 and add to row 3:
			\begin{equation*}A=
				\begin{pmatrix}[ccc|c]
					1&1&-1&9\\0&1&0.75&-0.75\\0&6&-8&58
				\end{pmatrix}
			\end{equation*}
		\item Multiple row 2 by 6 and subtract from row 3
			\begin{equation*}A=
				\begin{pmatrix}[ccc|c]
					1&1&-1&9\\0&1&0.75&-0.75\\0&0&-12.5&62.5
				\end{pmatrix}
			\end{equation*}
		\item Divide row 3 by -12.5:
			\begin{equation*}A=
				\begin{pmatrix}[ccc|c]
					1&1&-1&9\\0&1&0.75&-0.75\\0&0&1&-5
			\end{pmatrix}
			\end{equation*}
		\item Multiply row 3 by 0.75 and subtract from row 2:
			\begin{equation*}A=
				\begin{pmatrix}[ccc|c]
					1&1&-1&9\\0&1&0&3\\0&0&1&-5
			\end{pmatrix}
			\end{equation*}
		\item Subtract row 2 and add row 3 to row 1:
			\begin{equation*}A=
				\begin{pmatrix}[ccc|c]
					1&0&0&1\\0&1&0&3\\0&0&1&-5
			\end{pmatrix}
			\end{equation*}		
	\end{enumerate}
	Thus leaving the solution vector to be:
	\begin{equation*}
	\boxed{
		\begin{pmatrix} 
		x\\y\\z
		\end{pmatrix}
		=
		\begin{pmatrix}
		1\\3\\-5
		\end{pmatrix}}
	\end{equation*}
	\item Use of Cramer's Rule requires computation of the determinate of multiple matrices and division to compute the solution:
\begin{equation}
	\begin{pmatrix}
		x\\y\\z
	\end{pmatrix}
	=
	\begin{pmatrix}
		\dfrac{det(\boldsymbol{A})}{det(\boldsymbol{A_x})} \\
		\dfrac{det(\boldsymbol{A})}{det(\boldsymbol{A_y})} \\
		\dfrac{det(\boldsymbol{A})}{det(\boldsymbol{A_z})} \\
	\end{pmatrix}
	\label{eq:sol}
\end{equation}
Where $\boldsymbol{A_x}$, $\boldsymbol{A_y}$, and $\boldsymbol{A_z}$ represent a combination of the $\boldsymbol{A}$ and $\boldsymbol{b}$ comprised by replacing the associated vector in $\boldsymbol{A}$ with $\boldsymbol{b}$. 
\begin{equation*}
	\boldsymbol{A_x}=
	\begin{pmatrix}
		9&1&-1\\-6&8&6\\40&4&-6
	\end{pmatrix}
\end{equation*}
\begin{equation*}
	\boldsymbol{A_y}=
	\begin{pmatrix}
		1&9&-1\\0&-6&6\\-2&40&-6
	\end{pmatrix}
\end{equation*}
\begin{equation*}
	\boldsymbol{A_x}=
	\begin{pmatrix}
		1&1&9\\0&8&-6\\-2&4&40
	\end{pmatrix}
\end{equation*}
With the determinate being found using equation \ref{eq:det}.
\begin{equation}
	det(\boldsymbol{A})= a_1 \begin{vmatrix} b_2 & b_3 \\ c_2 & c_3\end{vmatrix}
	    -a_2 \begin{vmatrix} b_1 & c_1 \\ b_1 & c_3\end{vmatrix}
	   +a_3\begin{vmatrix} b_1 & b_2 \\ c_1 & c_2\end{vmatrix}
	\label{eq:det}
\end{equation}
Evaluation of the four determinates yields:
\begin{align*}
	det(\boldsymbol{A})=-100\\
	det(\boldsymbol{A_x})=-100\\
	det(\boldsymbol{A_y})=-300\\
	det(\boldsymbol{A_z})=500\\
\end{align*}
Substituting into equation \ref{eq:sol} yields:
\begin{equation*}
	\boxed{
	\begin{pmatrix}
		x\\y\\z
	\end{pmatrix}
	=
	\begin{pmatrix}
	1\\3\\-5
	\end{pmatrix}}
\end{equation*}

\item Finding the inverse is done by augmenting the identiy matrix onto $\boldsymbol{A}$ and reducing $\boldsymbol{A}$ to the identity. The augmented matrix is:
\begin{equation*}
\begin{pmatrix}[ccc|ccc]
	1&1&-1&1&0&0\\0&8&6&0&1&0\\-2&4&-6&0&0&1
\end{pmatrix}
\end{equation*}
The matrix inverse is found with the following steps:
\begin{enumerate}
	\item Divide row 2 by 8
	\begin{equation*}
		\begin{pmatrix}[ccc|ccc]
			1&1&-1&1&0&0\\0&1&0.75&0&0.125&0\\-2&4&-6&0&0&1
		\end{pmatrix}
	\end{equation*}
	\item Add 2 times row 1 to row 3
	\begin{equation*}
		\begin{pmatrix}[ccc|ccc]
			1&1&-1&1&0&0\\0&1&0.75&0&0.125&0\\0&6&-8&2&0&1
		\end{pmatrix}
	\end{equation*}	
	
	\item Subtract 6 times row 2 from row 3, then divide by -12.5
	\begin{equation*}
		\begin{pmatrix}[ccc|ccc]
			1&1&-1&1&0&0\\0&1&0.75&0&0.125&0\\0&0&1&-0.16&0.06&-0.08
		\end{pmatrix}
	\end{equation*}	

	\item Subtract 0.75 times row 3 from row 2
	\begin{equation*}
		\begin{pmatrix}[ccc|ccc]
			1&1&-1&1&0&0\\0&1&0&0.12&0.08&0.06\\0&0&1&-0.16&0.06&-0.08
		\end{pmatrix}
	\end{equation*}	
	\item Add row 3 and subtract row 2 from row 1
	\begin{equation*}
		\begin{pmatrix}[ccc|ccc]
			1&0&0&0.72&-0.02&-0.14\\0&1&0&0.12&0.08&0.06\\0&0&1&-0.16&0.06&-0.08
		\end{pmatrix}
	\end{equation*}
	\end{enumerate}
Thus $\boldsymbol{A}^{-1}$ is:
\begin{equation*}
	\begin{pmatrix}
		0.72&-0.02&-0.14\\0.12&0.08&0.06\\-0.16&0.06&-0.08
	\end{pmatrix}
\end{equation*}
From this $\boldsymbol{x}$ can be found by $\boldsymbol{A}^{-1}\boldsymbol{b}$:
\begin{equation*}\boxed{
	\begin{pmatrix}
	x\\y\\z
	\end{pmatrix}=
	\begin{pmatrix}
		0.72&-0.02&-0.14\\0.12&0.08&0.06\\-0.16&0.06&-0.08
	\end{pmatrix}
	\begin{pmatrix}
		9\\-6\\40
	\end{pmatrix}=
	\begin{pmatrix}
	1\\3\\-5
	\end{pmatrix}
	}
\end{equation*}
\end{enumerate}
